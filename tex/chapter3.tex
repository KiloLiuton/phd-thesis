\chapter{Mean field theory for small world networks}

The previous chapter focused mainly on simulation results for ring-lattices and small-world networks. In this chapter we develop a mean
field (MF) theory for the WCM dynamics on a system of $N$ oscillators coupled according to small world networks as described by the
Watts Strogatz algorithm in section~\ref{smallworld}. We start by re-stating the dynamics rules and proceed to take the continuous
limit.

In the adopted convention, superscripts denote an oscillators index (from 1 to $N$), and subscripts denote phase state (from 1 to 3).
Given a graph with nodes representing phase oscillators labeled by $x$, the unit at position $x$ is described by its phase value

\begin{align}
    \phi^x = 2\pi (s^x-1)/3 \\
    s^x \in \{1,2,3\} \notag
\end{align}

\noindent where $s^x$ is the state of unit $x$. For a given unit at $x$, the stochastic rate of transition from state $s^x\equiv j$ to
$j+1$ is written as $\phi^x_j \to \phi^x_{j+1}$, and is given by the Arrhenius form:

\begin{equation}
    \gamma^x_j = \omega^x\exp\left[ a\frac{n^x_{j+1} - n^x_j}{n^x} \right] \qquad x=1,\dots, N
    \label{rate}
\end{equation}

\noindent where $\omega^x$ is the natural frequency of the considered unit, $n^x_j$ is the count of how many of its neighbors are in
state $j$, and $n^x$ is its total number of neighbors.

Let the probability of finding the unit at $x$ in state $j$ be denoted by $p^x_j$. Since there are only three possible states we have
the restrictions $p^x_1+p^x_2+p^x_3=1$ for all $x$. The rate of change of the probabilities can be written in terms of the transition
rates $\gamma^x_j$, giving us $2N$ equations of motion:

\begin{align}
    \ddt{p}^x_1 &= \gamma^x_3(1 - p^x_1 - p^x_2) - \gamma^x_1 p^x_1 & \notag\\
    \ddt{p}^x_2 &= \gamma^x_1p^x_1 - \gamma^x_2 p^x_2 &x=1,\dots,N
    \label{eqmotion}
\end{align}

\noindent The first terms in the RHS of (\ref{eqmotion}) represent the fluxes of probability from oscillators in the previous state,
which can advance one phase state and increase the populations of oscillators in states 1 and 2 respectively. The second (negative)
terms represent the fluxes generated by the oscillator leaving the considered state, and thus reducing that states population.  This
dynamics occurs on small-world networks, which are generated following the Watts-Strogatz algorithm starting with a ring lattice.

A ring lattice is constructed by starting with a closed one-dimensional chain of $N$ phase oscillators, and than adding connections to
the first $K$ clockwise neighbors of each unit. The result of this process for $N=12$ and $K=2$ can be seen in figure~\ref{fig:ring} in
chapter~\ref{chap:article}.

To generate a small-world network from this base, the rewiring procedure considers each existing edge once, and with a probability $p$
it changes the clockwise vertex of this edge by a randomly selected node in the network. If the randomly selected node would create an
already existing edge or a self-connection, no action is performed.

\section{Mean field}

By defining the fractions inside the exponent of equation~(\ref{rate}) at a fixed time $t$ as

\begin{equation}
    \nu^x_j(t) \equiv \frac{n^x_j(t)}{n^x(t)}
\end{equation}

\noindent the transition rates are now written as $\gamma^x_j(t) = g\exp\left[ a(\nu^x_{j+1}(t) - \nu^x_j(t)) \right]$. The quantity $\nu^x_j$
describes the fraction of sites which are connected to node $x$ that are in state $j$ and thus also satisfies
$\nu^x_1+\nu^x_2+\nu^x_3=1$.

In the continuous limit, the average values of $\nu$ will dictate the dynamics. This average is taken over independent realizations of
the dynamics on different graphs produced by the rewiring process. We define the average value of $\nu$ as

\begin{align}
    \left< \nu^x_j(t) \right> = \frac{\left< n^x_j(t) \right>}{\left< n^x(t) \right>}
\end{align}

\noindent where the average is taken over independent realizations and \textit{not} over time.

By defining the dummy variable $D_{xx'}$, indicating the presence (or absence) of a connection between nodes $x$ and $x'$, we can write
the exact expressions for $n^x$ and $n^x_j$ in the discrete case. Let $D_{xx'}$ be defined as

\begin{align}
    D_{xx'} = 
    \begin{cases}
        1 \qquad &\text{if the connection $x$,$x'$ exists}\\
        0 \qquad &\text{otherwise}
    \end{cases}
\end{align}

\noindent then, if the states of sites $x,x'$ are $j,j'$ respectively, we have:

\begin{align}
    n^x_j &= \sum\limits_{x'=1}^ND_{xx'}\delta_{jj'}(t) \notag\\[8pt]
    n^x &= \sum\limits_{x'=1}^ND_{xx'}
\end{align}

\noindent where $\delta_{jj'}$ is the usual Kronecker delta defined by $\delta_{jj'}=\begin{cases}1 \quad j=j'\\0 \quad j\neq
j'\end{cases}$, and thus $\left< \nu^x_j(t) \right>$ becomes

\begin{align}
    \left< \nu^x_j(t) \right> &= \frac{\sum\limits_{x'=1}^N\left< D_{xx'}\delta_{jj'}(t)\right>}{\sum\limits_{x'=1}^N\left< D_{xx'} \right>}
    \notag \\[8pt]
    \left< \nu^x_j(t) \right> &= \frac{\sum\limits_{x'=1}^N\left< D_{xx'}\right>\left<\delta_{jj'}(t)\right>}{\sum\limits_{x'=1}^N\left< D_{xx'} \right>}
\end{align}

\noindent In order to solve the proposed MF we assume $\left< D_{xx'}\delta_{jj'}(t)\right> = \left<
D_{xx'}\right>\left<\delta_{jj'}(t)\right>$. In words this means the state of site $x'$ depends on the state of site $x$ regardless of
them being coupled or not. This is more accurate when the probability of rewiring $p$ is approximately zero, which is true for a large
number of networks in the small-world regime, as seen in figure~\ref{fig:small-world} of chapter~\ref{chap:article}. When $p=0$ we can
restrict the summation over the connected sites and recover the MF proposed by Escaff \textit{et al.}~\cite{escaff2014arrays}, making
$\delta_{jj'}$ truly independent from $D_{xx'}$.

The quantity $\left< \delta_{jj'}(t) \right>$ can be identified with the probability of finding the site at $x'$ in state $j$ at time
$t$, which we write as $P^{x'}_j(t)$. This probability is the central point of the mean-field

\begin{align}
    \left< \frac{D_{xx'}}{\sum_{x''=1}^ND_{xx''}}\delta_{jj'} \right> &= \left< \frac{D_{xx'}}{\sum_{x''=1}^ND_{xx''}} \right>\left< \delta_{jj'} \right> \notag\\[8pt]
    &= \left< \frac{D_{xx'}}{\sum_{x''=1}^ND_{xx''}} \right> P^{x'}_j
\end{align}
