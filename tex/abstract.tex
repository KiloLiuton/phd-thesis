In this work we extend the investigation of coupled discrete phase oscillators to circular networks with non-global coupling, using
analytic approximations and simulation. For the later, the chosen method is an event-driven simulation. Its general implementation
allows for the investigation of arbitrary graphs, and is used to investigate small-world networks. At the same time, a mean-field (MF)
approximation for small-world networks is introduced, which predicts the stability of travelling waves at positive coupling, where
usually the globally synchronized solution would be observed. In the zero disorder limit the MF recovers the mean-field approximation
proposed in previous works, but here travelling waves are found to be stable even when the underlying graph has some disorder.  The
wave solutions compete with global oscillations as well as with each other when there is more than one stable wave number, leading to
spontaneous transitions between wave numbers. Finite systems will always be subject to these fluctuations, but larger systems are more
robust since noise becomes smaller relative to wave period and amplitude.

Preliminary simulations and scaling analyses indicated that wave solutions did not lose stability if interaction range and system size
are increased in the same proportion, a property which is captured by the MF approximation. To further probe its validity, we tested
other predictions such as wave stability in the presence of disorder, the later introduced through rewiring the base graph. Another
finding is that the speed of propagation of such waves should increase with increasing natural frequencies, which is verified in
simulation.

\vspace{\onelineskip}
 
\noindent 
\textbf{Keywords}: Dissertation, Doctoral thesis, Dynamical systems, Statistical physics, Couple oscillators, Networks.
