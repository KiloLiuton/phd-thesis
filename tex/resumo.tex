\begin{otherlanguage*}{brazil}

Neste trabalho estendemos o estudo de sistemas de osciladores acoplados para redes circulares de acoplamento não global utilizando
modelagem matemática e simulações. O método simulacional é do tipo ``event driven'', e a sua implementação genérica permitiu a
investigação de grafos de conectividade arbitrários de modo que redes de mundo pequeno também puderam ser investigadas. Também
desenvolvemos uma teoria de campo médio para descrever o comportamento coletivo dos osciladores acoplados nessas redes de mundo
pequeno, que prevê a estabilidade de ondas viajantes em regimes de acoplamento positivo. Em geral nessas condições se observa uma
sincronia global entre osciladores, mas o alcance não global das interações permite a sobrevivência de tais estados. No entanto, a
descrição também prevê a estabilidade de ondas mesmo quando desordem de alcance global é introduzida no sistema através de um
algorítimo de reconexão do grafo subjacente. Nesse regime a solução de ondas viajantes está constantemente competindo com outros
estados estáveis, como oscilações globais ou ondas viajantes com número de onda diferente (quando há mais de um número de onda
estável). Isso faz com que o sistema nunca atinga propriedades macroscópicas estáveis, mesmo depois de longos tempos. Sistemas finitos
sempre estarão sujeitos a esse tipo de regime, com a criação e aniquilação espontânea de números de onda, mas se tornam mais estáveis
quando o tamanho do sistema cresce, tornando a intensidade das flutuações nas frentes de onda pequenas em relação a amplitude das
ondas.

Inicialmente, as simulações indicaram que soluções de ondas viajantes permanecem estáveis se o tamanho do sistema cresce
proporcionalmente ao alcance das interações, uma propriedade que foi capturada pela aproximação de campo médio. Para fortalecer a
validade de tal aproximação, algumas de suas previsões foram testadas, como a estabilidade de ondas viajantes na presença de desordem
introduzida pelo processo de reconexão. Outro resultado é a modulação da velocidade de propagação das ondas através de vieses
macroscópicos introduzidos nas frequências naturais de oscilação das unidades microscópicas do sistema, que também foi verificado em
simulação.

\vspace{\onelineskip}
 
\noindent 
\textbf{Palavras-chave}: Dissertação, Tese doutorado, Sistemas dinâmicos, Física estatística, Osciladores acoplados, Redes.
\end{otherlanguage*}{brazil}
