\chapter{Simulation behavior and results}
\label{chap:simulation}

In this chapter we present various simulations performed with the objective of both gaining insight into the behavior of oscillators
governed by such dynamics and also to test novels cases predicted by the MF description derived in chapter \ref{chap:mf}. These were
carried out at the Statistics and Computation Service (SCS) cluster at UMICH as well as local office and personal computers at UMICH
and UFMG.

\section{Simulation methods}

The procedure to simulate the dynamics described by the discrete master equations \ref{eq:motion} starts with some arbitrary base graph
that defines the coupling between oscillators. Each node in the graph represents an oscillator while the edges represent coupling with
constant strength $a$. The next step is giving every oscillator an initial phase. Since there are only three phase values, each one is
encoded as an integer: $0 \rightarrow 0$, $\pi/3 \rightarrow 1$ and $2\pi/3 \rightarrow 2$.

Given a graph
We simulate the dynamics described by the discrete master equations \ref{eq:motion} by performing a sequential update on one oscillator
at a time. 


\section{Critical couplings on small-world networks}

\subsection{Disorder favors global synchronization}

\section{Spontaneous wave creation and annihilation}
