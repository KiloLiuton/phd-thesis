\chapter*{Introduction}

In 1951 the Belousov-Zhabotinsky class of chemical reactions were created for the first time in a laboratory. This feat marked an
upheaval in thermodynamics, which had now been shown, experimentally, to be ruled by non-equilibrium dynamics even for long mixing
times. This generated for the first time interest in the field of coupled oscillators by the broader scientific community. In the
decades that followed, other areas of research independently led to the same interests. In 1958, Norbet Weiner at MIT was looking at
power spectra from brain scans and decided that there must be some kind of synchronization amongst coupled, periodically-firing,
neurons. Also inspired by many biological rhythms observed in nature, Arthur Winfree published in 1967 a paper on the mathematical
modelling of interacting oscillators. Winfree defined a phase oscillator as an abstraction of any system that possesses periodicity. As
such, a complicated periodic process inside a living cell could be described by a single real number, between $0$ and $2\pi$, and its
rate of change. The major simplification comes from the fact that when two such phase oscillators interact, each influences
\textit{only} the rate of change in phase of the other. Their trajectories in phase space are assumed to remain closed loops, and thus
the only effective change can be modeled as a change in the speed at which they traverse it.

Seven years after Winfree's 1967 paper, the young Japanese scientist Yoshiki Kuramoto recognized what made Winfree's model hard to
solve. He proposed a modified version of it, assuming a particular form for the coupling between oscillators. This led to an arguably
less realistic model, but one which Kuramoto was cunningly able to solve. This modified version, together with its solution, would
later be recognized as the Kuramoto model, becoming one of the major founding stones in the field.

Since then, a plethora of novel coupled oscillator models have been put forward. One class deals with discrete phase oscillators, in
which the phase that describes a unit is related to a finite set of possible values. Discrete-phase models are used to describe systems
which exhibit markedly distinct states. An example of such a system is interacting neurons, where one can identify the three distinct
phases of a neuron as firing, refractory and ready-to-fire.

Another element present in some models is stochasticity in the dynamics, which is deterministic in the Kuramoto model. The element of
stochasticity may be included to account for noise, such as in models of open systems that interact with a heat bath.

Here we consider discrete-phase models with three possible phase states, and stochasticity plays a role in the rate of change of the
phase state of each unit, being the probability of ``jumping'' to the next state per unit time. This model was introduced by Kevin
Wood, C.  Van den Broeck, R. Kawai and Katja Lindenberg \cite{Wood06a}, with the main motivation of finding the simplest possible
dynamics between phase oscillators that still led to a phase transition to synchrony. Initial investigations focused mainly on square
lattices and all-to-all connections, exposing the ubiquity of this type of self-organization. Often times, networks of coupled
oscillators observed in nature possess neither of these two. Neurons, fireflies and cardiac cells are all examples of systems that do
not operate on lattices or complete-graph networks. One common type of interaction network observed in real systems is a mix between
local and global interactions. Various algorithms have been proposed for the generation of such intermediate connectivity graphs, some
of which generate small-world networks, characterized by the presence of long-range interactions while preserving strong local
clustering.

A particular example for the generation of small-world networks is the Watts-Strogatz algorithm, which randomly breaks connections on
an existing graph and re-creates them between random vertices. When applied to certain starting graphs with strong local clustering,
the result is a small-world network. Here we investigate how the dynamics of the three-phase oscillator model unfolds on such networks,
via simulation and a mean-field analysis.
