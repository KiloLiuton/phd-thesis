\pretextual


% ---
% Folha de rosto
% (o * indica que haverá a ficha bibliográfica e deverá ser usado para teses 
%   e dissertações. Para projetos, favor retirar o *)
% ---
\imprimirfolhaderosto*
% ---

% ---
% Inserir a ficha bibliografica
% ---

%  A ficha catalografica, preparada pela biblioteca do departamento de física da UFMG é
% um elemento obrigatório para teses e dissertações. Inclua o arquivo pdf fornecido de acordo
% com o comando abaixo, modificando o nome e a localização do arquivo quando necessário.
%
\begin{fichacatalografica}
    \includepdf{fig/ficha-catalografica.pdf}
\end{fichacatalografica}

% ---
% Inserir folha de aprovação
% ---

% A folha de aprovação é um elemento obrigatório da versão final do trabalho e deverá ser incluído
% conforme comando abaixo.
%
% TODO: na versao final incluir a ficha de aprovacao
\begin{folhadeaprovacao}
     \includepdf{fig/folhadeaprovacao.pdf}
\end{folhadeaprovacao}
%


% ---
% Dedicatória - OPCIONAL
% ---
%\begin{dedicatoria}
%   \vspace*{\fill}
%   \centering
%   \noindent
%   \textit{\textbf{Dedicatória. Elemento opcional. Não há formatação a ser seguida.}\\
%    Este trabalho é dedicado às crianças adultas que,\\
%   quando pequenas, sonharam em se tornar cientistas.} \vspace*{\fill}
%\end{dedicatoria}
% ---

% ---
% Agradecimentos
% ---
\begin{agradecimentos}

I acknowledge the orientation of professor Ronald Dickman, who helped formalize mathematical arguments, select a simulation methodology
and more.

I would also like to acknowledge Professor Kevin Wood, from the University of Michigan for hosting me as an invited researcher,
providing computational resources as well as insightful conversations.

I acknowledge the partial funding provided by Capes and Cnpq.

\end{agradecimentos}
% ---

% ---
% Epígrafe - OPCIONAL
% ---
%\begin{epigrafe}
%    \vspace*{\fill}
%	\begin{flushright}
%		\textit{\textbf{Epígrafe. Elemento opcional. Não há formatação a ser seguida.}\\
%		``Não vos amoldeis às estruturas deste mundo, \\
%		mas transformai-vos pela renovação da mente, \\
%		a fim de distinguir qual é a vontade de Deus: \\
%		o que é bom, o que Lhe é agradável, o que é perfeito.\\
%		(Bíblia Sagrada, Romanos 12, 2)}
%	\end{flushright}
%\end{epigrafe}
% ---

% ---
% RESUMOS
% ---

% resumo em inglês
\begin{resumo}[Abstract]
In this work we extend the investigation of coupled discrete phase oscillators to circular networks with non-global coupling, using
analytic approximations and simulation. For the later, the chosen method is an event-driven simulation. Its general implementation
allows for the investigation of arbitrary graphs, and is used to investigate small-world networks. At the same time, a mean-field (MF)
approximation for small-world networks is introduced, which predicts the stability of travelling waves at positive coupling, where
usually the globally synchronized solution would be observed. In the zero disorder limit the MF recovers the mean-field approximation
proposed in previous works, but here travelling waves are found to be stable even when the underlying graph has some disorder.  The
wave solutions compete with global oscillations as well as with each other when there is more than one stable wave number, leading to
spontaneous transitions between wave numbers. Finite systems will always be subject to these fluctuations, but larger systems are more
robust since noise becomes smaller relative to wave period and amplitude.

Preliminary simulations and scaling analyses indicated that wave solutions did not lose stability if interaction range and system size
are increased in the same proportion, a property which is captured by the MF approximation. To further probe its validity, we tested
other predictions such as wave stability in the presence of disorder, the later introduced through rewiring the base graph. Another
finding is that the speed of propagation of such waves should increase with increasing natural frequencies, which is verified in
simulation.

\vspace{\onelineskip}
 
\noindent 
\textbf{Keywords}: Dissertation, Doctoral thesis, Dynamical systems, Statistical physics, Couple oscillators, Networks.

\end{resumo}

% resumo em português
\setlength{\absparsep}{18pt} % ajusta o espaçamento dos parágrafos do resumo
\begin{resumo}[Resumo]
\begin{otherlanguage*}{brazil}

Neste trabalho estendemos o estudo de sistemas de osciladores acoplados para redes circulares de acoplamento não global utilizando
modelagem matemática e simulações. O método simulacional é do tipo ``event driven'', e a sua implementação genérica permitiu a
investigação de grafos de conectividade arbitrários de modo que redes de mundo pequeno também puderam ser investigadas. Também
desenvolvemos uma teoria de campo médio para descrever o comportamento coletivo dos osciladores acoplados nessas redes de mundo
pequeno, que prevê a estabilidade de ondas viajantes em regimes de acoplamento positivo. Em geral nessas condições se observa uma
sincronia global entre osciladores, mas o alcance não global das interações permite a sobrevivência de tais estados. No entanto, a
descrição também prevê a estabilidade de ondas mesmo quando desordem de alcance global é introduzida no sistema através de um
algorítimo de reconexão do grafo subjacente. Nesse regime a solução de ondas viajantes está constantemente competindo com outros
estados estáveis, como oscilações globais ou ondas viajantes com número de onda diferente (quando há mais de um número de onda
estável). Isso faz com que o sistema nunca atinga propriedades macroscópicas estáveis, mesmo depois de longos tempos. Sistemas finitos
sempre estarão sujeitos a esse tipo de regime, com a criação e aniquilação espontânea de números de onda, mas se tornam mais estáveis
quando o tamanho do sistema cresce, tornando a intensidade das flutuações nas frentes de onda pequenas em relação a amplitude das
ondas.

Inicialmente, as simulações indicaram que soluções de ondas viajantes permanecem estáveis se o tamanho do sistema cresce
proporcionalmente ao alcance das interações, uma propriedade que foi capturada pela aproximação de campo médio. Para fortalecer a
validade de tal aproximação, algumas de suas previsões foram testadas, como a estabilidade de ondas viajantes na presença de desordem
introduzida pelo processo de reconexão. Outro resultado é a modulação da velocidade de propagação das ondas através de vieses
macroscópicos introduzidos nas frequências naturais de oscilação das unidades microscópicas do sistema, que também foi verificado em
simulação.

\vspace{\onelineskip}
 
\noindent 
\textbf{Palavras-chave}: Dissertação, Tese doutorado, Sistemas dinâmicos, Física estatística, Osciladores acoplados, Redes.
\end{otherlanguage*}{brazil}

\end{resumo}

% ---
% O SUMARIO É OBRIGATÓRIO
% ---

% ---
% inserir o sumario
% ---
\pdfbookmark[0]{\contentsname}{toc}
\tableofcontents*
\cleardoublepage
% ---


