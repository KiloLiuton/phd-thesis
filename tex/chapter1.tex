\chapter{Kuramoto Model}
\label{chap:kuramoto}


In this chapter we describe the Kuramoto model, its dynamics and self-consistency arguments that allow for a synchronous phase to
emerge at some critical coupling. To do this, an order parameter is introduced together with a mean-field (MF) approximation. The
stability analysis of the solutions of the Kuramoto model took decades to develop and led to some very anti intuitive findings, such as
the fact that the disordered phase for weak coupling is not stable, but neutrally stable, and is related to Landau damping in plasma
physics.  Nonetheless, the two branches of solutions identified in Kuramoto's initial work don't require solving the original set of
equations, but rather rely on assuming macroscopic behaviors for the system and tracing back what implications they have for the
microscopic dynamics.  This feedback from macro to micro is central to Kuramoto's self-consistency arguments, which helped popularize
his model by lowering the barrier of entry to understanding synchronization in complex systems. Much of the discussion developed in
this section are reproductions from the notes of da Fonseca Abud\cite{da_Fonseca_2018}.


\section{Definition}

The Kuramoto model consists of a collection of phase oscillators (``units'') which are all coupled to each other. We start with a
collection of $N$ oscillators and then take the continuous limit with $N \to \infty$. Each unit $j$ at time $t$ is described by its
phase value $\phase_j(t)$ and its natural frequency $\nfreq_j$, which is its rate of change in phase in isolation, and is assumed to be
constant. The natural frequencies $\vec{\nfreq}\equiv \{\nfreq_1,\dots,\nfreq_N\}$ are assumed to follow some frequency distribution
$\vec{\nfreq} \sim g(\vec{\nfreq})$ which is symmetric about some frequency $\bar{\nfreq}$, and unimodal. While the rate of change of
an isolated unit is time independent, the coupling between oscillators makes $\ddt{\phase_j}$ a function of the entire phase
configuration $\vec{\phase}(t) \equiv \{ \phase_1(t), \dots, \phase_N(t)\}$.

The biologically inspired Winfree model states that, at every instant, each oscillator sends a signal of strength $P(\phase)$ and
responds by adjusting its own phase value proportionally to $Q(\phase)$ when receiving signals from other units, which is captured in
the equation:

\begin{equation}
    \ddt{\phase_j} = \nfreq_j + \frac{1}{N}\sum^{N}_{k}P(\phase_k)Q(\phase_j), \qquad j=1,...,N
    \label{winfreeeq}
\end{equation}

In the special case where the product $P(\phase_k)Q(\phase_j)$ depends only on the phase difference between oscillators $j$ and $k$,
the product inside the summation can be written as a single function called the kernel $F$: $P(\Delta \phase)Q(\Delta \phase) \equiv
F(\Delta \phase)$. If furthermore the kernel is proportional to the sine function, then Winfree's equation~(\ref{winfreeeq}) becomes
the Kuramoto model:

\begin{equation}
    \ddt{\phase_j} = \nfreq_j + \frac{\kc}{N}\sum^{N}_{k}\sin(\phase_k - \phase_j), \qquad j=1,...,N
    \label{kura}
\end{equation}

\noindent where $\kc \geq 0$ is a real constant which describes the coupling strength between any pair of oscillators.

\section{Mean-field and order parameter}

To determine if the microscopic collection of phases represents a synchronized macro state, we introduce the Kuramoto order parameter
$R$. It maps a phase configuration $\vec{\phase}$ to a magnitude by representing the phase values as complex numbers on the unit circle
and performing a complex summation:

\begin{align}
    R(\vec{\phase})e^{i\phi(\vec{\phase})} &\equiv \frac{1}{N} \sum^{N}_{k} e^{i\phase_k(t)}
    \label{kuramotoop}
\end{align}

\noindent where $i=\sqrt{-1}$ and $\phi(\vec{\phase})\in[0,2\pi)$ is the resulting polar angle of the summation. Thus $R( \vec{\phase})
\in [0,1]$ determines the overall macroscopic alignment between phase values. With this definition the equations of motion can be
re-written as:

\begin{align}
    \ddt{\phase_j} &= \nfreq_j + \frac{\kc}{N}\sum^{N}_{k}\sin(\phase_k - \phase_j) \notag \\
                   &= \nfreq_j + \kc \operatorname{Im}\left( \frac{1}{N}\sum^{N}_{k} e^{i(\phase_k-\phase_j)} \right) \notag \\
                   &= \nfreq_j + \kc R(\vec{\phase}) \operatorname{Im}\left( e^{i(\phi(\vec{\phase}) - \phase_j)} \right) \notag \\[18pt]
    \ddt{\phase_j} &= \nfreq_j + \kc R(\vec{\phase}) \sin ( \phi(\vec{\phase}) - \phase_j )
    \label{kuramf}
\end{align}

We see that the equations of motion are now written in terms of the mean fields $R(\vec{\phase})$ and $\phi(\vec{\phase})$. In
particular, if the fields are constant ($\ddt{R} = \ddt{\phi}=0$) then the equations of motion become decoupled and can be solved.

The collective oscillations observed in real systems tell us that the macroscopic steady-state solution of stable synchronized
oscillations should indeed have constant $R>0$ while $\phi$ increases linearly with time (the system rotates synchronously with
constant phase speed).  Thus, $\phi=\Omega t$ and both $R$ and $\Omega$ are constant. Even though $\phi$ itself is not constant, it
turns out that the rotational symmetry of equation~(\ref{kuramf}) allows us to decouple the equations by moving to a rotating frame of
reference with the following substitution:

\begin{align}
    \forall j: \quad \phase_j(t) &= \phaserot_j(t) + \Omega t \notag\\
    &\Downarrow \notag\\
    \ddt{\phaserot_j} + \Omega &= \nfreq_j - \kc R \sin ( \phaserot_j ) \notag\\[10pt]
    \ddt{\phaserot_j} &= (\nfreq_j-\Omega) - \kc R \sin ( \phaserot_j )
    \label{kurarot}
\end{align}

Another macroscopic behavior observed in nature is the more trivial one, with no synchronization at all. In this state the phase
configuration is evenly distributed over all possible values, leading to constant $R=0$, and is also described by
equation~(\ref{kurarot}). Inspection of this equation shows that when $|\nfreq_j - \Omega| \leq \kc R$ there is some phase value for
which $\ddt{\phaserot_j}=0$, and thus oscillators become phase-locked when attaining this value. When $|\nfreq_j - \Omega| > \kc R$,
the $\phaserot_j$ can never stop varying and will permanently drift. For a synchronized solution we then expect two groups of
oscillators:

\begin{itemize}
    \item \textbf{Phase-Locked} $|\nfreq_j - \Omega|\leq \kc R$: A group of oscillators with natural frequencies close enough to
        $\Omega$ will become phase locked and sync.
    \item \textbf{Drifting} $|\nfreq_j - \Omega| > \kc R$: Remaining oscillators with frequencies too far from the generated mean-field
        will remain drifting ahead or behind the synced group.
\end{itemize}

Since we postulate that $R$ is time independent, self-consistency requires that the phase distribution generated by the drifting
oscillators must be stationary, otherwise $R$ would not be constant.

\section{Continuous Limit}

In the continuous limit $N\to \infty$, the distribution of oscillator phases can be written as the joint probability density in phases
$\phaserot$ and natural frequencies $\nfreq$ as:

\begin{equation*}
	P(\phaserot_j, \nfreq_j,t) \to p(\phaserot, \nfreq,t) \qquad \phaserot \in [-\pi, \pi) \qquad \nfreq \in (-\infty, \infty)
\end{equation*}

\noindent For given values of $\phaserot$ and $\nfreq$, the evolution follows

\begin{equation}
	\ddt{\phaserot} = (\nfreq - \Omega) - \kc R \sin(\phaserot).
	\label{eqmotioncontinuous}
\end{equation}

\noindent The density of oscillators with phase $\phaserot$ is then given by the marginal distribution:

\begin{align*}
    n(\phaserot) = \myint_{-\infty}^{\infty} p(\phaserot, \nfreq,t) d\nfreq
\end{align*}

\noindent which by the definition of conditional probability gives:

\begin{align*}
    n(\phaserot) = \myint_{-\infty}^{\infty} p(\phaserot | \nfreq)g(\nfreq) d\nfreq
\end{align*}

\noindent where $p(\phaserot|\nfreq)$ is the conditional probability density that an oscillator with given natural frequency
$\nfreq$ has phase value $\phaserot \equiv \phaserot(t)$ at time $t$.

By using the subscripts $L$ and $D$ to indicate the \textit{phase-locked} and \textit{drifting} groups, we get the expression for the
total density:

\begin{gather}
    n(\phaserot) = n_L(\phaserot) + n_D(\phaserot) \notag\\[10pt]
    n(\phaserot) = \myint_{|\nfreq - \Omega|\leq \kc R} p_L(\phaserot|\nfreq,t) g(\nfreq) d\nfreq + \myint_{|\nfreq - \Omega| > \kc R} p_D(\phaserot|\nfreq,t) g(\nfreq) d\nfreq
    \label{density}
\end{gather}

The conditional probability for the phase-locked group can then be obtained by solving the equation of motion~(\ref{kurarot}), and for
the drifting group it can be derived by the self-consistency assumption that it must be a stationary distribution in order to have a
time-independent order parameter.

\section{Phase distribution of phase-locked oscillators}

For the phase locked group we know that for $t \to \infty$, $\ddt{\phaserot}=0$ for some value of $\phaserot^\star$ and $\nfreq$. Thus,
we solve equation~(\ref{eqmotioncontinuous}) to obtain:

\begin{gather}
    \phaserot^\star = \arcsin \left( \frac{\nfreq - \Omega}{\kc R} \right), \quad
    \phaserot^\star \in \left[ -\frac{\pi}{2}, \frac{\pi}{2} \right] \notag
\end{gather}

Therefore, the conditional distribution of $\phaserot$ given $\nfreq$ is a Dirac delta function centered around $\phaserot^\star$,
since this is an attractor for all oscillators with natural frequency $\nfreq$.

\begin{gather}
    \label{lockedprob}
    p_L(\phaserot | \nfreq) = \delta\left[ \phaserot - \arcsin \left( \frac{\nfreq - \Omega}{\kc R} \right) \right], \quad
    \phaserot \in \left[ -\frac{\pi}{2}, \frac{\pi}{2} \right]
\end{gather}

We now perform the left integral in equation~(\ref{density}) to find the density of phase locked oscillators to obtain:
\begin{align*}
    n_L(\phaserot) &= \myint_{\Omega - \kc R}^{\Omega + \kc R} \delta\left[ \phaserot - \phaserot^\star \right] g(\nfreq) d\nfreq
\end{align*}

\noindent but $\nfreq = \Omega + \kc R \sin\phaserot^\star$ and $d\nfreq = \kc R \cos \phaserot^\star d\phaserot^\star$ and thus

\begin{align}
    n_L(\phaserot) &= \myint_{-\frac{\pi}{2}}^{\frac{\pi}{2}} \delta \left[ \phaserot - \phaserot^\star \right] g(\Omega + \kc R \sin
    \phaserot^\star) \kc R \cos \phaserot^\star d\phaserot^\star \notag\\[10pt]
    n_L(\phaserot) &=
    \begin{cases}
        g(\Omega + \kc R \sin\phaserot) \kc R \cos\phaserot \quad &|\phaserot| \leq \frac{\pi}{2}\\
        0 & |\phaserot| > \frac{\pi}{2}
    \end{cases}
    \label{densitylocked}
\end{align}

Equation~(\ref{densitylocked}) shows us that the phase locked oscillators are distributed in the half-moon around the order parameter,
in accord with direct simulations of the equations of motion. Since the $\phaserot$ values are phase-locked, the probability
distribution of synchronized oscillators $n_L$ is also time independent, in addition to $n$.

\section{Phase distribution of drifting oscillators}

In order to derive the distribution of phases for drifting oscillators, for which $|\nfreq - \Omega| > \kc R$, consider an interval
$[\phaserot, \phaserot + \delta \phaserot ]$ in phase values. The fraction of oscillators with natural frequency $\nfreq \in [\nfreq',
\nfreq' + \delta\nfreq]$ inside this interval is given by:

\begin{align}
	\left[ \myint_{\phaserot}^{\phaserot + \delta \phaserot} p(\phaserot' | \nfreq) d\phaserot' \right] \delta \nfreq
	\label{count}
\end{align}

The change per unit time in the fraction of oscillators inside this interval is then given by the time derivative of~(\ref{count}), but
since all intervals endpoints $\nfreq$, $\delta\nfreq$ and $\phaserot$, $\delta\phaserot$ are fixed we have:

\begin{equation}
	\partial_t \left[ \myint_{\phaserot}^{\phaserot + \delta \phaserot} p(\phaserot' | \nfreq) d\phaserot' \right] =
	\myint_{\phaserot}^{\phaserot + \delta \phaserot}\partial_t p(\phaserot' | \nfreq) d\phaserot'
\end{equation}

\noindent Conservation of probability inside the interval $[\phaserot,\phaserot+\delta\phaserot]$ mandates that the net change in
probability must equal the difference between fluxes at the endpoints. The probability current is written as $j(\phaserot, \nfreq, t) =
p(\phaserot|\nfreq)\ddt{\phaserot}$

\begin{align}
	\myint_{\phaserot}^{\phaserot + \delta \phaserot}\partial_t p(\phaserot' | \nfreq) d\phaserot' = 
	\label{control_interval}
\end{align}

Intuitively, equation~(\ref{control_interval}) implies that the rate of change of probability inside the control interval is equal to
the difference of fluxes at the extremities of the interval. Dividing this expression by $\delta\phaserot$ and taking the limit of
$\delta\phaserot \to 0^+$ leads us to the continuity equation for the conservation of probability density of oscillators with natural
frequency $\nfreq$:

\begin{gather}
    \lim_{\delta\phaserot \to 0^+} \frac{p(\phaserot+\delta\phaserot | \nfreq)\ddt{\phaserot}(\phaserot+\delta\phaserot) - p(\phaserot | \nfreq)\ddt{\phaserot}(\phaserot)}{\delta\phaserot} = \partial_\phaserot \left[ p(\phaserot | \nfreq) \ddt{\phaserot}(\phaserot) \right]  \notag\\
    \lim_{\delta\phaserot \to 0^+} \partial_t \left[ \frac{1}{\delta\phaserot}\myint_{\phaserot}^{\phaserot + \delta \phaserot} p(\phaserot' | \nfreq) d\phaserot' \right] = \partial_t \left[ p(\phaserot | \nfreq) \right] \notag\\[10pt]
    \partial_t \left[ p(\phaserot | \nfreq) \right] - \partial_\phaserot \left[ p(\phaserot | \nfreq) \ddt{\phaserot}(\phaserot) \right] = 0
    \label{continuity}
\end{gather}

We have seen that requiring that the total phase distribution $n(\phaserot)$ be stationary implies that the distribution of
phase-locked oscillators $n_L(\phaserot)$ is also stationary. But since the distribution of drifting oscillators is given by
$n_D(\phaserot) = n(\phaserot) - n_L(\phaserot)$, it must also be stationary. Thus we have

\begin{gather}
    \partial_t [ n_D(\phaserot) ] = 0\notag\\
    \myint_{|\nfreq - \Omega| > \kc R} \partial_t [ p_D(\phaserot | \nfreq)] g(\nfreq) d\nfreq = 0
    \label{stationary_drift}
\end{gather}

Even though there are time-dependent forms for $p_D(\phaserot | \nfreq)$ that could satisfy equation~(\ref{stationary_drift}), the
assumption that it is indeed stationary ties up with the original assumption about the total density $n$ and the self consistency
argument, a central point in Kuramoto's original analysis. The condition $\partial_t [p_D(\phaserot | \nfreq)]=0$ for all $\phaserot$
thus imply, through the continuity equation~(\ref{continuity}), that $\partial_\phaserot [p_D(\phaserot |
\nfreq)\ddt{\phaserot}(\phaserot)]=0$ and so:

\begin{gather}
    p(\phaserot | \nfreq) \ddt{\phaserot}(\phaserot) = C(\nfreq) \notag\\
    \Downarrow \notag\\
    p_D(\phaserot | \nfreq) = \frac{C(\nfreq)}{(\nfreq - \Omega) - \kc R \sin \phaserot}
    \label{nD}
\end{gather}

In order to obtain the total density $n_D(\phaserot)$ we must marginalize the joint distribution $p_D(\nfreq | \nfreq)g(\nfreq)$. In
order to do that we need to find the value of $C$ as a function of $\nfreq$, $C(\nfreq)$. This can be done by using the normalization
condition $\int_{-\pi}^{\pi}p_D(\phaserot | \nfreq)d\phaserot=1$ which leads us to:

\begin{align}
    \frac{1}{C(\nfreq)} &= \myint_{-\pi}^{\pi}\frac{1}{\nfreq - \Omega - \kc R \sin \phaserot}d\phaserot
    \label{constantC}
\end{align}

\noindent Using equation~(2.551-3) from the integrals table~\cite{jeffrey2007table} with the condition $|\nfreq-\Omega| > \kc R$ we get
the solution:

\begin{align}
    a&=(\nfreq-\Omega) \qquad b=\kc R \notag\\[5pt]
    f(\phaserot) &= \arctan\frac{a\tan \frac{\phaserot}{2}+b}{\sqrt{a^2-b^2}} \notag\\[5pt]
    \myint\frac{1}{\pm a-b\sin \phaserot}d\phaserot &= \pm\frac{2}{\sqrt{a^2-b^2}}f(\phaserot) \qquad a^2 > b^2 \notag
\end{align}

\noindent which is positive for $\nfreq > \Omega + \kc R$ and negative for $\nfreq < \Omega - \kc R$. Substituting back into
equation~(\ref{constantC}) to obtain $C$ gives:

\begin{align}
    \frac{1}{C_\pm(\nfreq)} &= \pm\frac{2}{\sqrt{(\nfreq-\Omega)^2 - (\kc R)^2}} \left[ \lim_{\phaserot \to \pi^-} f(\phaserot) - \lim_{\phaserot \to (-\pi)^+} f(\phaserot) \right] \notag\\[10pt]
    C(\nfreq) &=
    \begin{cases}
        -\frac{\sqrt{(\nfreq-\Omega)^2 - (\kc R)^2}}{2\pi} \qquad &\nfreq < \Omega - \kc R \\[10pt]
        \frac{\sqrt{(\nfreq-\Omega)^2 - (\kc R)^2}}{2\pi} \qquad &\nfreq > \Omega + \kc R
    \end{cases}
    \label{constant}
\end{align}

Now, we use the expression for $C(\nfreq)$ to perform the integration in $\nfreq$ as defined in equation~(\ref{density}):

\begin{align}
    n_D(\phaserot) &= \myint_{-\infty}^{\Omega - \kc R} p_D(\phaserot | \nfreq)g(\nfreq)d\nfreq +
                      \myint_{\Omega + \kc R}^{\infty} p_D(\phaserot | \nfreq)g(\nfreq)d\nfreq \notag\\
                   &= \myint_{-\infty}^{\Omega - \kc R} \frac{C_-(\nfreq)}{(\nfreq - \Omega) - \kc R\sin\phaserot}g(\nfreq)d\nfreq +
                      \myint_{\Omega + \kc R}^{\infty} \frac{C_+(\nfreq)}{(\nfreq - \Omega) - \kc R\sin\phaserot}g(\nfreq)d\nfreq
                      \label{driftdensityintermediate}
\end{align}

To perform this integration we make use of the assumptions that $g$ is symmetric and unimodal. In addition, we also assume that the
frequency of global oscillations to which the system converges to, $\Omega$, is coincident with the point of symmetry of $g$. Namely,
we assume that $\Omega = \bar{\nfreq}$. With this assumption, $g(\Omega + x) = g(\Omega - x)$, allowing us to make the following change
of variables:

Substitute equation~(\ref{constant}) into (\ref{driftdensityintermediate}) and change variables $\nfreq -
\Omega = -x$ in the first integral and $\nfreq - \Omega = x$ in the second. Finally, we get the expression for the density of drifting
oscillators:

\begin{align}
    n_D(\phaserot) = \frac{1}{\pi}\myint_{\kc R}^\infty \frac{x\sqrt{x^2 - (\kc R)^2}}{x^2 - (\kc R\sin\phaserot)^2}g(x+\Omega)dx
    \label{densitydrift}
\end{align}


\section{Order Parameter}


In the rotating frame, the Kuramoto order parameter is given by the integral:

\begin{align}
    R = &\myint_{-\pi}^{\pi} e^{i\phaserot}n_L(\phaserot)d\phaserot + \myint_{-\pi}^{\pi} e^{i\phaserot}n_D(\phaserot)d\phaserot
\end{align}

From equation~(\ref{densitydrift}) we see that $n_D(\phaserot - \pi) = n_D(\phaserot)$, but $e^{i(\phaserot - \pi)} = -e^{i\phaserot}$,
and thus the rightmost integral is zero. The leftmost integral on the right hand side (RHS) can then be written with
equation~(\ref{densitylocked}) for the phase-locked oscillators.

\begin{align}
     R = &\kc R\myint_{-\pi/2}^{\pi/2} (\cos \phaserot + i \sin \phaserot ) \cos\phaserot g(\Omega + \kc R \sin\phaserot) d\phaserot
\end{align}

\noindent but since $g$ is symmetric around $\Omega$ we end up with

\begin{align}
     R = \kc R\myint_{-\pi/2}^{\pi/2} g(\Omega + \kc R \sin\phaserot) \cos^2\phaserot d\phaserot
     \label{finalkura}
\end{align}

Thus, we see that for a fixed value of the coupling strength $\kc$ there is always at least one solution for the order parameter $R$.
The trivial (disordered) solution is given by $R=0$. If $R$ is non-zero, then we can divide equation~(\ref{finalkura}) to obtain the
condition for the existence of the solution for that particular $\kc$:

\begin{align}
     \kc\myint_{-\pi/2}^{\pi/2} g(\Omega + \kc R \sin\phaserot) \cos^2\phaserot d\phaserot = 1
     \label{kc}
\end{align}

The simpler case of identical oscillators can be obtained from equation~(\ref{kc}) by setting $g(x) = \delta(x-\Omega)$ which results
in $\kc_c = 1$ as the critical coupling strength for such a network to attain synchronization.

For a general for of the natural frequency distribution $g$ we can obtain the critical coupling strength $\kc_c$ by expanding its
Taylor series around $\Omega$ and neglecting terms $\mathcal{O}(R)$ and higher. This is because when $\kc \to \kc_c^+$, then $R \to
0^+$ and so:

\begin{align}
    \frac{1}{\kc} &= \myint_{-\pi/2}^{\pi/2} \left[ g(\Omega) + g''(\Omega)\frac{(\kc R \sin\phaserot)^2}{2} + \mathcal{O}(R^3) \right] \cos^2\phaserot d\phaserot \\[10pt]
    \frac{1}{\kc_c} &= \lim_{R\to0^+} \myint_{-\pi/2}^{\pi/2} \left[ g(\Omega) + g''(\Omega)\frac{(\kc R \sin\phaserot)^2}{2} + \mathcal{O}(R^3) \right] \cos^2\phaserot d\phaserot
\end{align}

\noindent where $g'(\Omega)=0$ since $g(\Omega)$ is a maximum. Therefore the critical coupling is given by:

\begin{align}
    \kc_c = \frac{2}{\pi g(\Omega)}
\end{align}

As the coupling approaches the critical value from above, the order parameter scales as:

\begin{align}
    \frac{1}{\kc} &= \frac{1}{\kc_c} - \frac{\kc_c^2 R^2|g''(\Omega)| }{16} + \cancelto{0}{\mathcal{O}(R^3)} \notag\\
                  &\Downarrow \notag \\
    R(\kc) &= \frac{4}{\sqrt{\pi\kc_c^3|g''(\Omega)|}} \sqrt{\frac{\kc - \kc_c}{\kc_c}}
\end{align}

We see that the order parameter is sensitive to the curvature of $g$ at the maximum, growing ever more rapidly as $g$ becomes narrower.
In the limit of identical oscillators, and the order parameter becomes discontinuous at the transition.

