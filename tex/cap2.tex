\chapter{Introduction}

In 1951 the Belousov-Zhabotinsky class of chemical reactions were created for the first time in a laboratory. They would become a
classic example of non-equilibrium thermodynamics and helped instigate interest in the sub-area of complex systems: coupled
oscillators. In the decades that followed other areas of research led, independently, to the same interest in studying coupled
oscillators systems. In 1958, Norbet Weiner at the MIT was looking at power spectrum graphs generated from brain scans and decided that
there must be some kind of synchronization amongst many coupled, periodically-firing, neurons. Also inspired by the many biological
rhythms observed in nature, Arthur Winfree published in 1967 a paper on the mathematical modelling of interacting oscillators. Winfree
defined a phase oscillator as an abstraction of a system that possesses a closed loop in its phase-space. A single real number between
$0$ and $2\pi$ could describe, for example, a living cell that undergoes any kind of periodic cycle. The major simplification comes
from the fact that when two such phase oscillators interact, each influences {\textit only} the phase value of the other. Their
trajectories in phase space are assumed to be fixed, and only the velocity with which they traverse it is influenced by the coupling.

Seven years after Winfree's paper from 1967, the then young Japanese scientist Yoshiki Kuramoto was inspired by it. He recognized what
made Winfree's model hard to solve and proposed an arguably less realistic version of it, but the beauty of the now possible
mathematical solution made up for it. So much so that it got it's own name, the Kuramoto model, one of the major founding stones of
modern investigations on these kinds of dynamical systems.

This work could be considered as one such branch, where the phase value that describes a unit has been discretized into only three
possible values, $0$, $2\pi/3$ and $4\pi/3$. These types of discrete phase models saw much use in describing systems which present
distinct states. One such system is interacting neurons, where one could choose to represent three different neuronal states as firing
($\varphi_0$), refractory ($\varphi_1$) and ready-to-fire ($\varphi_2$). Another phenomenon explained by one such model is the
synchronous blinking of some firefly species, which had long been reported but scientists were skeptical of it until camera recorders
got around. If $n$ state representations are desired than the phase value of each unit would be one of $\{\varphi_0 ...
\varphi_{n-1}\}$ with $\varphi_{n}=\varphi_0$.
% TODO: Why was stochasticity introduced to these models?


\chapter{Kuramoto Model}


In this chapter we describe the Kuramoto model and reproduce his self-consistency arguments that allow for a synchronous phase to
emerge at some critical coupling. In order to do this an order parameter must be introduced together with a mean-field approximation.
The stability analysis of the solutions of the Kuramoto model took decades to develop and led to some very anti-intuitive findings such
as the fact that the disordered phase for low-couplings is not stable, but neutrally stable, and is related to Landau damping in plasma
physics. These later discussions are not in the scope of this chapter.

\section{Equation}

The Kuramoto model deals with an infinite number of phase oscillators which are all coupled to each other. We start with a collection
of $N$ oscillators and then take the limit $N \to \infty$. Each unit is described by its phase value $\varphi_i$ and its natural
frequency $\omega_i$. The natural frequency is the frequency with which an isolated oscillator rotates. The population of oscillators
are assumed to have some distribution $g(\omega)$ which is symmetric about some frequency $\omega_0$ and has a single maximum at this
same value.

The biologically inspired Winfree model states that at every instant, each oscillator sends a signal of strength $P(\varphi)$ and
responds by adjusting its own phase value by an amount $R(\varphi)$, which is captured in the equation:

\begin{equation}
    \label{winfreeeq}
    \dot{\varphi_j} = \omega_j + \frac{1}{N}\sum_{k=1}^{N}P(\varphi_k)R(\varphi_j), \qquad i=1,...,N
\end{equation}

To make this solvable, we say that the angular frequency of a unit depends only on its relative phase to others. Furthermore, this
dependency is through a sine kernel. This will make possible the self consistency arguments done later. This set of assumptions
transforms (\ref{winfreeeq}) into:

\begin{equation}
    \label{kuramotoeq}
    \dot{\varphi_j} = \omega_j + \frac{K}{N}\sum_{k=1}^{N}sin(\varphi_k - \varphi_j), \qquad i=1,...,N
\end{equation}

\noindent where $K>0$ is a real constant which describes the coupling strength and  $N>>1$.

\section{Order Parameter}

\section{Fubar}

\subsection{Fufoobar}

Fubbar

\subsection{Fubfoo}

f... BAR!

\subsection{F}

.

\chapter{Fooblusions}

b

